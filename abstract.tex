\documentclass[11pt,a4paper]{article}

\usepackage[margin=1in]{geometry}
\usepackage{url}
\usepackage{hyperref}
\usepackage{natbib}

\title{Neural dialogue act tagging with transformer pre-training (or something like that...)}

\author{
  First Author \\
  Affiliation / Address line 1 \\
  Affiliation / Address line 2 \\
Affiliation / Address line 3 \\
  {\tt email@domain} \\\and
  Second Author \\
  Affiliation / Address line 1 \\
  Affiliation / Address line 2 \\
  Affiliation / Address line 3 \\
  {\tt email@domain} \\
}
 

\date{}

\begin{document}
\maketitle

\begin{abstract}
  abstract
\end{abstract}

\section{Introduction}
\begin{itemize}
  \item \cite{austinHowThingsWords2009} -- Speech acts
  \item \cite{coreCodingDialogsDAMSL1997} -- DAMSL dialogue act scheme
  \item Recent transformer models, such as BERT, adopt the strategy of pre-training a neural network
    on large corpora of unlabeled data with a language modeling task, 
    and then fine-tuning an output layer on labeled data
    They achive state-of-the-art results in various NLP token- and sentence-level tasks,
    including (...) \cite{devlinBERTPretrainingDeep2018}.
    In this work, we investigate whether these pre-trained models can be adapted for dialogue tasks.
    In particular, we have two questions:
    1. Are pre-trained transformer encoding (BERT) helpful for representation in a dialogue task (speficially, dialogue act tagging)?
    2. Are additional dialogue specific features helpful (discourse markers, disfuency, laughter)?
\end{itemize}

\section{Related Work}
\begin{itemize}
  \item \cite{cerisaraEffectsUsingWord2vec2017} -- 
    Use LSTM with and without word2vec initialization to encode discourse utterances and predict dialogue act tags.
    No notion of discourse context.
    Word2vec does not improve dialogue act recoginition, possibly becasue it does not capture lexical-semantic information relevant to dialogue.

  \item \cite{khanpourDialogueActClassification} -- 
    Uses RNN and stacked LSTM with word2vec embeddings 
    to encode utterances.
    No notion of discourse context.

  \item \cite{pragstVectorRepresentationUtterances2018} --
    Uses a skip thought approach to learn utterance representations by predicting context utterances.

  \item \cite{kalchbrennerRecurrentConvolutionalNeural2013} --
    Encode utterances (sentences) convolution-based composition of the sequence of words (HCNN). 
    Use an RNN with agent-specific weights to produce utterance label predictions.
    Also talk about the hidden layer of the RNN as representing discourse context. 

  \item \cite{tranPreservingDistributionalInformation2017} --
    Propogates a distribution over dialogue acts instead of greedily choosing a lable at each step, 
    a kind of discourse context.

  \item \cite{stolckeDialogueActModeling2000} - 
    HMM dialogue act tagging

  %\item \cite{sordoniNeuralNetworkApproach2015} - generate utterances taking into account previous countext
\end{itemize}

\section{Model}
The proposed model has two components: an utterance encoder, and a sequence model that predicts the dialogue act tag.
For the sequence model, we use a simple RNN.
Conceptually the hidden state of this RNN represents the discourse context which, 
together with the encoded utterance, is used to predict the discourse act tag.

\section{Experiments}
Our aim is to test the effectiveness of different utterance encoding schemas. 
\begin{enumerate}
  \item Average word embedding (word2vec)
  \item CNN (with/without word2vec initalization)
  \item BERT encoded sentences (with/without additional unsupervised pretraining)
  \item Average BERT word embeddings
\end{enumerate}

\bibliography{abstract}{}
\bibliographystyle{acl_natbib}

\end{document}
