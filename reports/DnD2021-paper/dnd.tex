\documentclass[twoside,11pt]{article}

% Any additional packages needed should be included after dnd.
% Note that dnd.sty includes epsfig, amssymb, natbib and graphicx,
% and defines many common macros, such as 'proof' and 'example'.
%
% It also sets the bibliographystyle to plainnat; for more information on
% natbib citation styles, see the natbib documentation, a copy of which
% is archived at http://www.jmlr.org/format/natbib.pdf

\usepackage{dnd}
\usepackage{times}



% Heading arguments are {volume}{year}{pages}{submitted}{accepted}published online}{author-full-names}{doi-number}

\dndheading{issue(number)}{year}{firstpage--lastpage}{Name1 Surname1, Name2 Surname2, and Name3 Surname3}{10.5087/dad.DOINUMBER}

% Short headings should be running head and authors last names

\ShortHeadings{Dialogue Acts and Higher Order Structure}{{Di Eugenio}, Xie and Serafin}
\firstpageno{1}



%



% # Strand 1. Laughter and dialogue acts
% - (BN) Experiment 1, 2 (+L), 3 (+L)
% - ensure that "x" DAs are removed
% - PCA: DAs with and w/o L side-by-side
% - (?) AMI and SWBD mappings and comparison

% # Strand 2. Laughter prediction
% - (VM) re-do some experiments from Maraev et al. 2019
% - predict whether next utterance contains laughter, also predict suf/pre/infix or standalone
% - joint training with DAR (or pre-training?)
% - human evaluation
% - (VM) think of appropriate background, e.g. from Tian (?)

% # Strand 3. "Non-verbal" DAs
% - (BN) remove "x" from target classes for testing 
% - PCA?
% - (VM) background on non-verbals


\begin{document}


\title{Dialogue \& Discourse Article Template}

\author{\name Name1 Surname1 \email email@author1.com \\
       \addr Affiliation1\\
       Affiliation1 second line
       \AND
       \name Name2 Surname2 \email email@author2.com \\
       \addr Affiliation2\\
       Affiliation2 second line
       \AND 
       \name Name3 Surname3  \email email@author3.com\\
       \addr Affiliation3\\
       Affiliation3 second line}

\editor{Name Surname}
\submitted{MM/YYYY}{MM/YYYY}{MM/YYYY}

\maketitle

\begin{abstract}%
 Abstract from {\em Dialogue Act Classification, Higher Order Dialogue Structure, and
 Instance-Based Learning} by Barbara di Eugenio, Zhuli Xie, and Riccardo Serafin: 
  The main goal of this paper is to explore the predictive power of
  dialogue context on Dialogue Act classification, both as concerns 
  the linear context 
  provided by previous dialogue acts, and the hierarchical context 
  specified  by conversational
  games. As our learning approach, we  extend Latent Semantic
  Analysis (LSA) as Feature LSA (FLSA), and combine FLSA with
  the k-Nearest Neighbor algorithm. FLSA adds
  richer linguistic features to LSA, which only uses words.   
   The k-Nearest Neighbor algorithm obtains its best results when
  applied to the reduced semantic spaces generated by FLSA.
  Empirically, our results are better than previously published
  results on two different corpora, MapTask and CallHome
  Spanish. Linguistically, we confirm and extend previous observations
  that the hierarchical dialogue structure encoded via the notion of
  game is of primary importance for dialogue act recognition.
\end{abstract}

\begin{keywords}
keyword1, keyword2, keyword3, keyword4, keyword5
\end{keywords}



\section{Introduction}

%%bibliographystyle is included in dnd

\bibliography{...}

\end{document}
